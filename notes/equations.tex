\documentclass[letterpaper,12pt]{paper}


%% Graphics and math
\usepackage{graphics,graphicx,subfigure}
\usepackage{amsmath}
\usepackage{amsfonts}
\usepackage[makeroom]{cancel}

\newcommand{\lnrho}{\ensuremath{\ln \rho}}
\newcommand{\fast}{\ensuremath{\epsilon_{ijk}}}
\def\dbar{{\mathchar'26\mkern-12mu d}}

\title{Notes on fully compressible, stratified convection}
\author{Jeff Oishi}
\begin{document}
\maketitle
\section{Basic Equations}
\label{sec:equations}


We want the fully compressible in a form suitable for describing
low-Mach number convection in a stratified, static background
atmosphere. This atmosphere could be a polytrope, but it could also be
specified by a 1D stellar or planetary model.

We begin with the basic equations of hydrodynamics: continuity,
Navier-Stokes, energy conservation, and an equation of state. For now,
we will limit ourselves to an ideal gas equation of state. These basic
equations can be written in tensor notation, using the Einstein
summation convetion as

\begin{subequations}
  \label{eq:basic_eqns}
  \begin{equation}
    \partial_t \rho + \partial_i (\rho u_i) = 0\\
  \end{equation}
  \begin{equation}
    \partial_t u_i + u_k \partial_k u_i = -\frac{\partial_i p}{\rho} +
    g_i + \frac{\partial_k \sigma_{ik}}{\rho}\\
  \end{equation}
  \begin{equation}
    \partial_t T + u_i \partial_i T = - (\gamma - 1) T \partial_i u_i +
    \frac{\sigma_{ik}}{c_v \rho} \partial_k u_i + \frac{\partial_i
      (K \partial_i T)}{c_v \rho}
  \end{equation}
  \begin{equation}
    P = R \rho T,
  \end{equation}
\end{subequations}
where $R = c_p - c_v$, $\gamma = c_p/c_v$, $g_i = -g \delta_{iz}$ is the acceleration
due to gravity, $K$ is the heat conductivity, and $\sigma_{ik}$ is the
stress tensor, defined below.

The vast majority of these derivations follows Landau \& Lifschitz.

\subsection{Stress Tensor}
\label{sec:stress}

The stress tensor can be constructed from a few simple
arguments. Viscosity can only act between parcels of fluid in
\emph{relative} motion; thus the stress tensor must be first order in
$\partial_k u_i$ with no $u_i$ terms. It must be symmetric
(e.g. proportional to $\partial_i u_k + \partial_k u_i$) to avoid
containing motions corresponding to solid body rotations,
$u_i = \fast \Omega_j r_k$. The most general tensor that satisfies
these constraints is
\begin{equation}
  \label{eq:general_tensor}
  \sigma_{ik} = a (\partial_i u_k + \partial_k u_i) + b \delta_{ik} \partial_l u_l, 
\end{equation}
but we write it as 
\begin{equation}
  \label{eq:stress}
  \sigma_{ik} = \mu \left(\partial_i u_k + \partial_k u_i - \frac{2}{3} \partial_l u_l \delta_{ik}\right) + \xi \delta_l u_l \delta_{ik},
\end{equation}
where $\mu$ is the dynamic shear viscosity, and $\xi$ is the bulk
viscosity. This form is the sum of a traceless term proportional to
$\mu$\footnote{remember that $\delta_{ii} = 3$, not 1!} and a purely
diagonal term proportional to $\xi$. We will assume here that $\xi$.
vanishes. We will sometimes use $\sigma^{'}_{ik}$, defined by
$\sigma_{ik} = \mu \sigma^{'}_{ik}$.

\subsection{Variables \& Parameters}
\label{sec:variables}

Our dynamical variables are $u_i = (u,v,w)$, $\lnrho$, $P$, and $T$,
though we eliminate $P$ via the equation of state in the developed
version. Typically, we parameterize dissipation by the kinematic
viscosity $\nu = \mu /\rho$ and $K$. $K$ is the \emph{heat} diffusion
coefficient, where $F_T = K \nabla T$ is the total heat flux in the
absence of bulk motions. 

$\nu$ is a dissipation coefficient with dimensions
$\left[ L^2/T\right]$, while $[K] = \left[ L^2 E/ T \Theta\right]$,
where $\Theta$ is a temperature unit and $E$ is an energy unit. We may
also want to use the dynamic viscosity,
$\mu = \rho \nu = \left[ M /L T\right]$. We may also want to
parameterize thermal diffusion via the temperature dissipation
coefficient, $\chi = K/c_p \rho = \left[ L^2/T\right]$.

We will hold either $\chi$ or $K$ constant and either $\nu$ or $\mu$
constant. How these choices affect the results of calculations is a
secondary objective of our study.


\subsection{Continuity}
\label{sec:continuity}

For the continuity equation expressing the conservation of mass, our
only modification is to change the variable from $\rho$ to $\lnrho$.

Using the fact that $\partial_t \lnrho = \partial_t \rho/\rho$, we can
substitute the standard continuity equation for $\partial_t \rho =
-\rho \partial_k u_k - u_k \partial_k \rho$ to arrive at 
\begin{equation}
  \label{eq:lnrho_continuity}
  \partial_t \lnrho + u_k \partial_k \lnrho + \partial_k u_k = 0.
\end{equation}

\subsection{Navier-Stokes}
\label{sec:ns}
We modify the Navier-Stokes equation by explicitly substituting the
$P$ term using the equation of state and converting $\rho$ to
$\lnrho$. This has implications for the diffusion term, and we will
see when we decompose the variables into their mean and fluctuating
parts that this will introduce non-linear diffusion terms. There are
two places where $\rho$ appears in the Navier-Stokes equation.

First, we tackle the pressure term, $\partial_i P/\rho$. We note that
$\partial_i P/\rho = \partial_i (P/\rho) + P \partial_i
\rho/\rho^2$.
Substituting in the ideal gas equation of state, $P = R \rho T$, we
see that the first term is $R \partial_i T$ if we assume the
composition of the gas (and thus $R$) is fixed. The second term can be
rewritten as $P \partial_i \lnrho/\rho$, into which we can again
substitute the equation of state, leaving $R T \partial_i
\lnrho$. Thus, 
\begin{equation}
  \label{eq:press_term}
  -\frac{\partial_i P}{\rho} = -R \partial_i T - R T \partial_i \lnrho.
\end{equation}

The other $\rho$ term is in the viscous term, $\partial_k
\sigma_{ik}/\rho$. First, again we apply the chain rule, $\partial_k
(\sigma_{ik}/\rho) = \partial_k \sigma_{ik}/\rho -
\sigma_{ik} \partial_k \lnrho/\rho$. Next, we expand $\sigma_{ik} = \mu
\sigma^{'}_{ik}$, and write $\sigma_{ik} \partial_k \lnrho/\rho = \nu
\sigma^{'}_{ik} \partial_k \lnrho$. This leaves us with 
\begin{equation}
  \label{eq:visc_term}
  \frac{\partial_k \sigma_{ik}}{\rho} = \partial_k (\nu
  \sigma^{'}_{ik}) + \nu \sigma^{'}_{ik} \partial_k \lnrho
\end{equation}
\subsection{Temperature}
\label{sec:temp}
The temperature equation is really nothing other than equation of
internal energy, which in turn comes from the total conservation of
energy. We must account for the viscous heating and the diffusion of
temperature, and this in turn requires us to understand the entropy
gains from these processes. We also will arrive at a form in which we
may want to specify the temperature diffusivity $\chi$, although the
background state may be defined in terms of the \emph{heat flux} $F_{T,i} =
K \partial_i T$. There are a number of curious points along this route
which I will remark on. Again, this follows mostly from Landau \&
Lifschitz, with a bit from Choudhuri thrown in.

Let us begin at the beginning: the first law of thermodynamics, which
tells us 
\begin{equation}
  \label{eq:1LOT}
  \dbar q = d\epsilon + P d \left(\frac{1}{\rho}\right),
\end{equation}
where $q$ is the amount of heat entering per unit mass, $\epsilon$ is
the internal energy per unit mass, $P$ is pressure, and $\rho$ is
density. $1/\rho$ is sometimes referred to as the specific volume of
the gas. We can put this in terms of state variables by
$ds \equiv \dbar q/T$. 

For an ideal gas, $\epsilon = c_v T$ and $P  = R \rho T$. Using
equation~(\ref{eq:1LOT}), we can divide by $dt$ to form the Lagrangian
derivative, $dX/dt = \partial_t X + u_k \partial_k X$. This gives
\begin{equation}
  \label{eq:heat}
  \frac{dq}{dt} = \frac{d\epsilon}{dt} - \frac{P}{\rho^2} \frac{d\rho}{dt},
\end{equation}
which we can combine with the continuity
equation. Equation~(\ref{eq:basic_eqns}a) can be written in terms of
the Lagrangian derivative of $\rho$, giving $d\rho/dt =
-\rho \partial_i u_i$. Substituting this, and rearranging we arrive at
\begin{equation}
  \label{eq:int_en}
  \frac{d\epsilon}{dt} = - \frac{P}{\rho} \partial_i u_i + \frac{dq}{dt}.
\end{equation}
In this form it is easy to see that the internal energy $\epsilon$ of
a given parcel of fluid only changes due to compression (that is,
negative divergence) or from heat flowing into the fluid. 

Each of these terms must come at the expense of some other part of the
overall energy budget if the fluid is thermally isolated. Thermal
isolation is, of course, a terrible approximation for much of
astrophysical fluid dynamics: often the cooling rate
$\mathcal{L} = - dq/dt$ is a dominant driver of dynamics, especially
in the ISM. However, here we will assume that it is a good
approximation. I believe, though I have not yet proved that this will
be true whenever we are studying the dynamics well below the
photosphere of an object. In such a case, we can probably assume that
radiation carries energy like a conduction term.

The first law (equation~(\ref{eq:1LOT})) is essentially a statement of
conservation of \emph{thermal} energy. Of course, in a fluid, we need
a stronger statement: the \emph{total} energy must be conserved. In
the case of stratified fluids, there are three energy reservoirs whose
sum must be constant: thermal, kinetic, and gravitational
potential. While above we used the \emph{specific} (that is, per unit
mass) formulation, now we will use the \emph{density} (that, per unit
volume) formulation. Thus, we write the total energy density
$E = \rho \epsilon + \rho u^2/2 + \rho g z$, which has dimension $\left[ M
  L^2/T^2/L^3 \right] = \left[ M / L T^2 \right]$. We can write this in the form
of a conservation law,
\begin{equation}
  \label{eq:energy_conservation}
  \partial_t E  + \partial_i \mathcal{F}_{Ei} = 0,
\end{equation}
where $\mathcal{F}_{Ei}$ is the energy flux in the $i$
direction. We can state without proof that the flux is 
\begin{equation}
  \label{eq:energy_flux}
  \mathcal{F}_{Ei} = \rho u_i \left(\frac{u^2}{2} + h \right) -
  K \partial_i T,
\end{equation}
where $h = \epsilon + P/\rho$ is the enthalpy. This is a somewhat
curious equation. The second term in the heat flux is simply Fourier's
law, where $K$ is the conductivity. One might reasonably suppose that
the flux of \emph{total} energy might simply be the advection of that energy
in the ideal limit $K \rightarrow 0$. However, when the enthalpy is
added in, there is an additional term in the flux: $u_i P$. This is
easy to understand if we integrate both sides of the equation over
volume, to see the evolution of the total energy:
\begin{equation}
  \label{eq:total_energy}
  \partial_t \int \left(\frac{\rho u^2}{2} + \rho \epsilon \right) dV =
  - \int \partial_i \left[ \rho u_i \left( \frac{u^2}{2} + \epsilon + \frac{P}{\rho} \right)
  \right] dV.
\end{equation}
Using divergence theorem, we can change the flux integral on the right
hand side to 
\begin{equation}
  \label{eq:flux_int}
  - \int \partial_i \left[ \rho u_i \left( \frac{u^2}{2} + \epsilon + \frac{P}{\rho} \right)
  \right] dV  = - \oint \rho u_i \left( \frac{u^2}{2} + \epsilon + \frac{P}{\rho} \right)
  df_i,
\end{equation}
where $df_i$ is the area normal to $x_i$. This surface integral can be
further split into two terms:
\begin{equation}
  \label{eq:surf_int}
    - \oint \left[ \rho u_i \left( \frac{u^2}{2} + \epsilon + \frac{P}{\rho} \right)
  \right] df_i = -\oint \rho u_i \left( \frac{u^2}{2} + \epsilon
  \right) df_i - \oint P u_i df_i.
\end{equation}
The first term is simply the advection of energy by the velocity field
$u_i$. Recalling that work is $W = \int F_i dx_i$ (where $F_i$ is a
force vector), the second term is clearly the work done on the surface
of the fluid by the pressure. This justifies the inclusion of enthalpy
$h$, rather than simply internal energy density $\rho \epsilon$, in
the energy flux $\mathcal{F}_{Ei}$.

Using equation~(\ref{eq:int_en}) and the ideal gas internal energy
relation, $\epsilon = c_v T$, we can write the temperature form of the
internal energy equation,
\begin{equation}
  \label{eq:temp_eq}
  c_v (\partial_t T + u_k \partial_k T)  = -\frac{P \partial_i
    u_i}{\rho} + \frac{dq}{dt}.
\end{equation}
For an adiabatic fluid, $dq/dt = 0$, and the temperature equation is
complete. However, a viscous fluid cannot be adiabatic, as the
dissipation of kinetic energy by viscosity is irreversible and thus
must raise the temperature $T$ and the entropy per unit mass $s$ of
the fluid.

In order to develop the terms corresponding to irreversible,
non-adiabatic processes, we make use of the total energy equation, and
the definition of entropy. In short, what we will do is construct a
kinetic energy equation, an entropy equation, and then show that the
losses from kinetic energy show up in the \emph{adiabatic} internal
energy by reversible processes and in the entropy equation by
irreversible processes. The right hand side of the entropy equation is
simply $T^{-1} dq/dt$. With that in hand, we can complete
equation~(\ref{eq:temp_eq}).

The first step is to note that kinetic energy density
$E_{kin} = \rho u^2/2$. We can then construct an equation for its time
evolution using the Navier-Stokes equation, continutity, and yet
another application of the chain rule:
$\partial_t (\rho u^2)/2 = \rho u_i \partial_t u_i + u \partial_t
\rho/2$. Substituting from the continuity equation and the Navier-Stokes equations, 
\begin{equation}
  \label{eq:kin_en}
  \partial_t E_{kin} = -u_i \partial_i P - \rho u_i u_k \partial_k u_i + \rho g_i u_i + u_i \partial_k \sigma_{ik} - \frac{u^2}{2} \partial_k ( \rho u_k).
\end{equation}
Next, we want to write this in conservation form
$\partial_t E_{kin} = - \nabla \cdot \mathcal{F}_{E_{kin}} +
\mathcal{L}_{E_{kin}}$.
We'll call terms that contribute to $\mathcal{F}_{E_{kin}}$ ``flux
terms,'' and terms contributing to $\mathcal{L}_{E_{kin}}$ ``loss
terms,'' regardless of their sign. First, we need to rewrite the
second and fifth terms on the right hand side of
equation~(\ref{eq:kin_en}) as
\begin{equation}
  \label{eq:kin_en_term2}
  - \rho u_i u_k \partial_k u_i = -\frac{\partial_k (\rho u_k u^2)}{2} + \frac{\rho u^2 \partial_k u_k}{2} + \frac{u_k u^2 \partial_k \rho}{2}, 
\end{equation}
and
\begin{equation}
  \label{eq:kin_en_term3}
  -\frac{u^2}{2} \partial_k(\rho u_k) = -\frac{\rho u^2 \partial_k u_k }{2} -\frac{u^2 u_k \partial_k \rho}{2}.
\end{equation}
Both terms from equation~(\ref{eq:kin_en_term3}) cancel with the
latter two in equation~(\ref{eq:kin_en_term2}), leaving the first term, which is in flux form. Next, we expand the pressure and viscous terms,
\begin{equation}
  \label{eq:kin_en_pressure}
  -u_i \partial_i P  = - \partial_i (P u_i)+ P \partial_i u_i,
\end{equation}
and
\begin{equation}
  \label{eq:kin_en_viscous}
u_i \partial_k \sigma_{ik} = \partial_k (u_i \sigma_{ik}) - \sigma_{ik} \partial_k u_i.
\end{equation}
Again, there are flux terms and loss terms. Combining all of these expansions and cancellations, we arrive at a final form for the kinetic energy density evolution,
\begin{equation}
  \label{eq:final_kin_en}
  \partial_t E_{kin} = - \partial_k \left[ u_k P + \frac{\rho u^2 u_k}{2} - u_i \sigma_{ik} \right] + p \partial_i u_i - \sigma_{ik} \partial_k u_i - \rho g u_z,
\end{equation}
so the loss term is
$\mathcal{L}_{E_{kin}} = p \partial_i u_i - \sigma_{ik} \partial_k u_i
+ \rho g_i u_i$.
These terms must appear in other forms of energy. The first two appear
as thermal energy; the latter will appear in the gravitational potential energy budget.

In a constant gravitational field with acceleration
$g_i = -g \delta_{iz}$, we can define the gravitational energy density
just like we would in high school, $E_g = \rho g z$. Thus, the
evolution equation is 
\begin{equation}
  \label{eq:grav_en}
  \partial_t E_g = \partial_t (\rho g z) = g z \partial_t \rho.
\end{equation}
Substituting from the continuity equation, and noting that $\partial_k
z = \delta_{iz}$, we end up with
\begin{equation}
  \label{eq:grav_en_final}
  \partial_t E_g = \rho g u_z - \partial_k (\rho g u_k z).
\end{equation}

To complete our argument, we formulate a total energy equation,
\begin{equation}
  \label{eq:total_energy}
  \partial_t E = - \partial_i \mathcal{F}_{tot,i},
\end{equation}
where $E = \rho u^2/2 + \rho \epsilon + \rho g z$.

Because we neglect radiative losses from the bulk of the fluid, there
are no source nor sink terms, though the boundary conditions will
allow for energy fluxes into and out of the box. We can write the
total energy flux as 
\begin{equation}
  \label{eq:total_flux}
  \mathcal{F}_{tot,i} = \rho u_i \left( \frac{u^2}{2} + h + gz\right) - u_k \sigma_{ki} - K \partial_i T,
\end{equation}
where again the enthalpy $h = \epsilon + \frac{P}{\rho}$. 

We expand the left hand side of equation~(\ref{eq:total_energy}) as
$\partial_t (\rho \epsilon) + \partial_t (\rho u^2/2) + \partial_t
(\rho g z)$, and use the results from our previous decompositions of
$E_{kin}$ (equation~\ref{eq:final_kin_en}) and $E_{g}$
(equation~(\ref{eq:grav_en_final}) to arrive at
\begin{equation}
  \label{eq:total_thermal}
\partial_t E = \partial_t (\rho \epsilon) - \partial_k \left[ u_k P + \frac{\rho u^2 u_k}{2} -
  u_i \sigma_{ik} \right] + p \partial_i u_i - \sigma_{ik} \partial_k
u_i - \rho g u_z + \rho g u_z - \partial_k (\rho g u_k z).
\end{equation}
Now, we use the chain rule on $\partial_t (\rho \epsilon) =
\rho \partial_t \epsilon + \epsilon \partial_t \rho$ and then use
continuity to get 
\begin{equation}
  \label{eq:thermal_terms}
  \rho \partial_t \epsilon - \epsilon \partial_i (\rho u_i) =
  \rho \partial_t \epsilon + \rho u_i \partial_i \epsilon - \partial_i
  (\rho u_i \epsilon) = \rho d_t \epsilon - \partial_i (\rho u_i \epsilon),
\end{equation}
where $d_t \epsilon$ is the Lagrangian derivative of the internal
energy. Substituting equation~(\ref{eq:thermal_terms}) into
equation~(\ref{eq:total_thermal}) and collecting flux terms, we arrive
at
\begin{equation}
  \label{eq:nearly_final_energy}
\partial_t E = \rho d_t \epsilon - \partial_k \left[ \rho u_k
  \epsilon u_k +  P + \frac{\rho u^2 u_k}{2} \rho g u_k z -
  u_i \sigma_{ik} \right] + p \partial_i u_i - \sigma_{ik} \partial_k
u_i.
\end{equation}
To complete our task, we simply add and subtract $\partial_i
(K \partial_i T)$ to the right hand side. Then the $\partial_t E$ on the
left and the $\partial_k [\ldots ]$ term on the right sum to zero,
leaving a residual equation for $\epsilon$:
\begin{equation}
  \label{eq:residual_eps}
  \rho d_t \epsilon + p \partial_i u_i = \partial_k u_i \sigma_{ik}
  + \partial_i (K \partial_i T).
\end{equation}
Once again, we invoke the first law (equation~(\ref{eq:1LOT})) in
entropy form, and we arrive at 
\begin{equation}
  \label{eq:final_entropy}
  \rho T d_t s = \partial_k u_i \sigma_{ik}
  + \partial_i (K \partial_i T).
\end{equation}
A simple reshuffling, plus the identity $T ds = \dbar q$ gives us 
\begin{equation}
  \label{eq:heat_dt}
  \partial_t q = \frac{\partial_k u_i \sigma_{ik}}{\rho}
  + \frac{\partial_i (K \partial_i T)}{\rho},
\end{equation}
which can be substitued into equation~(\ref{eq:temp_eq}) to arrive at
our final temperature equation,
\begin{equation}
  \label{eq:final_temperature}
    \partial_t T + u_i \partial_i T  = -(\gamma - 1) T \partial_i
    u_i + \frac{\partial_k u_i \sigma_{ik}}{c_v \rho}
  + \frac{\partial_i (K \partial_i T)}{c_v \rho}.
\end{equation}
We also note that we can write
\begin{equation}
  \label{eq:square_sigma}
  \partial_k u_i \sigma_{ik} = \frac{\mu}{2} \left(\partial_k u_i
    + \partial_i u_k - \frac{2}{3} \delta_{ik} \partial_l u_l
  \right)^2,
\end{equation}
and in the next section, we will use this form to write out the
actual, as-implemented version.

\section{Mean-field decomposition}
\label{sec:meanfield}
Because of our interest in convection and g-modes, all of which occurs
on timescales much shorter than the secular evolution of the stellar
or planetary structure, it is advantageous to split the background
into a mean radial structure and a 3D fluctuation. Note that there is
no approximation here; we make no assuption that the fluctuation is
small. We will denote the pressure, for example, as
$P = P_0(z) + P_1(x,y,z,t)$. $P_0$ is really a function of spherical
radius, which in our cartesian boxes corresponds to $z$. We write the
velocity in terms of its components as
$\mathbf{u} = u \hat{x} + v \hat{y} + w \hat{z}$. The density is
decomposed \emph{after} taking its natural log:
$\lnrho = (\lnrho)_0 + (\lnrho)_1$; we will write this without
the parentheses.

We will write all of our equations Dedalus-style: with linear terms on
the left hand side, non-linear terms on the right, and all in terms of
first derivatives. In order to construct the first-order formulation,
we first provide some auxiliary variables:
\begin{equation}
  \label{eq:aux_u}
  \partial_z u + u_z = 0
\end{equation}

\begin{equation}
  \label{eq:aux_v}
  \partial_z v + v_z = 0
\end{equation}

\begin{equation}
  \label{eq:aux_w}
  \partial_z w + w_z = 0
\end{equation}

\begin{equation}
  \label{eq:aux_u}
  \partial_z T_1 + T_{1z} = 0
\end{equation}

\subsection{Continuity}
\label{sec:continuity_mfd}

We begin the mean field decomposition with continuity, which is straightforward:
\begin{equation}
  \label{eq:continuity_mfd}
  \partial_t \lnrho_1 + u_z \partial_z \ln rho_0 + \partial_x u
  + \partial_y v + w_z =
  - u_x \partial_x \lnrho_1   - v \partial_y \lnrho_1   - w \partial_z \lnrho_1.
\end{equation}

\subsection{Navier-Stokes}
\label{sec:ns_mfd}

The Navier-Stokes equation comes in two forms, with constant $\nu$ and
constant $\mu$. First we consider constant $\mu$. Here, we write each
component separately.
\begin{equation}
  \begin{aligned}
    \label{eq:ns_x_mfd}
    \partial_t u & + R \partial_x T_1 + R T_0 \partial_x \lnrho_1 - \nu
    \left[ \partial_x^2 u + \partial_y^2 u + \partial_z u_z +
      \frac{1}{3}\partial_x \nabla \cdot u \right] \\
& - \nu \partial_z \lnrho_0 \left(\partial_x w + u_z \right)  \\
    & = - u \partial_x u - v \partial_y u - w v_z - R T_1 \partial_x \lnrho_1 \\
    & + \nu \left[ \left( 2 \partial_x u - \frac{2}{3} \nabla \cdot
        u\right) \partial_x \lnrho_1 + (\partial_x v + \partial_y
      u)\partial_y \lnrho_1 + (\partial_x w + \partial_x u) \partial_z \lnrho_1 \right]\\
\end{aligned}
\end{equation}

\begin{equation}
  \begin{aligned}
    \label{eq:ns_y_mfd}
    \partial_t v & + R \partial_y T_1 + R T_0 \partial_y \lnrho_1 - \nu
    \left[ \partial_x^2 v + \partial_y^2 v + \partial_z v_z +
      \frac{1}{3} \partial_y \nabla \cdot u \right] \\
& - \nu \partial_z \lnrho_0 \left(\partial_y w + v_z \right)  \\
    & = - u \partial_x v - v \partial_y v - w v_z - R T_1 \partial_y \lnrho_1 \\
    & + \nu \left[ (\partial_y u + \partial_x
      v)\partial_x \lnrho_1 + \left( 2 \partial_y v - \frac{2}{3} \nabla \cdot
        u\right) \partial_y \lnrho_1 + (\partial_y w + v_z) \partial_z \lnrho_1 \right]\\
\end{aligned}
\end{equation}

\begin{equation}
  \begin{aligned}
    \label{eq:ns_z_mfd}
    \partial_t w & + R \partial_z T_0 + R \partial_z T_1 + R
    T_0 \partial_z \lnrho_0 + R T_0 \partial_z \lnrho_1 + R
    T_1 \partial_z \lnrho_0 + g \\
&- \nu
    \left[ \partial_x^2 w + \partial_y^2 w + \partial_z w_z +
      \frac{1}{3} \partial_z \nabla \cdot u \right] - \nu \partial_z \lnrho_0 \left(2 w_z - \frac{2}{3} \nabla \cdot u  \right)  \\
    & = - u \partial_x w - v \partial_y w - w w_z - R T_1 \partial_z \lnrho_1 \\
    & + \nu \left[ (\partial_z u + \partial_x
      w)\partial_x \lnrho_1 + (\partial_z v + \partial_y w) \partial_y \lnrho_1  + \left( 2 w_z - \frac{2}{3} \nabla \cdot
        u\right) \partial_z \lnrho_1\right]\\
\end{aligned}
\end{equation}


\end{document}
